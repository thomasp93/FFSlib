\documentclass[a4paper, 10pt, italian]{article} % imposto i dati del documento
\usepackage[utf8]{inputenc} % font caratteri
\usepackage{babel} % attivo l'italiano
\author{Thomas Pertile} % definisco l'autore
\title{Domande di Linguaggi di Programmazione} % imposto il titolo
\begin{document}
\maketitle % inserisco il titolo nel testo

\section{Che cos’\`{e} una macchina astratta per un linguaggio di programmazione L?} % Filippo 
Dato un linguaggio di programmazione L, una macchina virtuale denotata con M_L, \`{e} ogni insieme di strutture dati e algoritmi che permettono la memorizzazione e l'esecuzione di programmi scritti in L. 

\section{Si descriva l’implementazione interpretativa pura di un linguaggio.} % Thomas
L'implementazione interpretativa pura di un linguaggio \`{e} quando l'interpreta M_L \`{e} implementato usando un set di istruzioni di Lo.

\section{Si descriva l’implementazione compilativa pura di un linguaggio.}

\section{Si descriva l’implementazione di un linguaggio mediante macchina intermedia.}

\section{Che cosa sono la grammatica, la semantica e la pragmatica di un linguaggio?}

\section{Che cosa sono le grammatiche libere?}

\section{Si descriva la notazione di Backus e Naur di una grammatica libera.}

\section{Che cosa sono le derivazioni e gli alberi di derivazione di una grammatica libera?}


\end{document}
